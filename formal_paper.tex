\documentclass{article}
\usepackage{amsmath, amssymb, amsthm, url, hyperref}
\usepackage[T1]{fontenc}
\usepackage[utf8]{inputenc}
\usepackage{cleveref}
\usepackage[thinc]{esdiff}

\newtheorem{Lemma}{Lemma}
\newtheorem{Theorem}{Theorem}

\newcommand{\simlt}{\mathrel{\underset{\sim}{<}}}

\let\epsilon\varepsilon

\begin{document}
	\section{Notation}
	\begin{itemize}
		\item Let $RV_\alpha^+$ denote eventually monotonic regularly varying functions with index $\alpha$ (see \cite{bingham1987regular}). We assume the domain and range are $\mathbb{R}_{>0}$
		\item Let $S(x)=\#\{(i, j)\mid i\le j, f(i)+f(i+1)+\dots+f(j)\le x\}$
		\item Let $s_k(x)=\#\{(i, j)\mid j-i=k, f(i)+f(i+1)+\dots+f(j)\le x\}$
		\item Let $\sim$ denote asymptotic equivalence. That is, 
		\[
		f(x)\sim g(x), x\to a\iff \lim_{x\to a} \frac{f(x)}{g(x)} = 1.
		\]
		We often omit the "$x\to a$" part, when it is clear from context.
		\item Denote $f(x)\lesssim g(x) \iff \limsup\frac{f(x)}{g(x)}\le1$ and similarly 
		\[f(x)\gtrsim g(x) \iff \limsup\frac{f(x)}{g(x)}\ge1\].
	\end{itemize}
	
	\section{Preliminaries}
	We assume throughout this section that $n=n(x)$ is the unique integer satisfying
	\[
	\sum_{i=1}^{n} f(i)\le x<\sum_{i=1}^{n+1} f(i).
	\]
	Where $f : \mathbb{R}_{>0}\to\mathbb{R}_{>0}$.
	\begin{Lemma}\label{sum f Asymp}
		Given $f\in RV_\alpha^+$ we have
		\[
		x\sim \frac{n^{\alpha+1}L(n)}{\alpha+1}
		\]
	\end{Lemma}
	\begin{proof}
		Let $\epsilon>0$ and split the sum
		\[
		\sum_{i=1}^{n} f(i)=\sum_{i=1}^{n\epsilon} f(i)+\sum_{i=n\epsilon}^{n} f(i).
		\]
		We will first examine the second sum.
		\[
		\sum_{i=n\epsilon}^{n} f(i)=\sum_{i=n\epsilon}^{n} i^\alpha L(i)\notag\\
		\]
		Using uniform convergence theorem (see Theorem 1.2.1 in \cite{bingham1987regular})
		\begin{align}
		\sum_{i=n\epsilon}^{n} i^\alpha L(i)&\sim L(n)\sum_{i=n\epsilon}^{n} i^\alpha\notag\\
		&\sim L(n)\int_{n\epsilon}^{n} t^\alpha dt\notag\\
		&=\frac{1}{\alpha+1}L(n)n^{\alpha+1}(1-\epsilon^{\alpha+1})\notag
		\end{align}
		Examining the first sum
		\[
		0<\sum_{i=1}^{n\epsilon} f(i)<M+n\epsilon f(n\epsilon),
		\]
		for some constant $M$. Since $f$ is eventually increasing
		\[
		M+n\epsilon f(n\epsilon)< M+n\epsilon f(n)=M+\epsilon n^{\alpha+1}L(n).
		\]
		Combining these we get
		\begin{align}
		\frac{1}{\alpha+1}n^{\alpha+1}L(n)(1-\epsilon^{\alpha+1})\lesssim\sum_{i=1}^{n} f(i)&\lesssim M+\frac{1}{\alpha+1}n^{\alpha+1}L(n)(1+\epsilon-\epsilon^{\alpha+1})\notag\\
		\sum_{i=1}^{n} f(i)&\sim \frac{1}{\alpha+1}n^{\alpha+1}L(n)\notag
		\end{align}
	\end{proof}
	
	\begin{Lemma}\label{sum f range asymp}
		Let $\epsilon>0$ given $k,l>0$ such that $k-l>\epsilon$ and $f\in RV_\alpha^+$. If \[\sum_{i=nl}^{nk} f(i)<x\] then
		\[
		\sum_{i=nl}^{nk} f(i)\sim \frac{1}{\alpha+1}L(n)n^{\alpha+1}(k^{\alpha+1}-l^{\alpha+1}).
		\]
	\end{Lemma}
	\begin{proof}
		From definition of $n$ it follows $k\le l+1$. Since $f$ is eventually increasing
		\[
		\sum_{i=nl}^{nk} f(i)>n(k-l) f(nl)>n\epsilon f(nl).
		\]
		Using Potter's bound (see Theorem 1.5.6 \cite{bingham1987regular}) and Lemma~\ref{sum f Asymp}
		\begin{align}
			\frac{1}{\alpha+1}n^{\alpha+1}L(n)&\gtrsim n\epsilon f(nl)\notag\\
			&= n^{\alpha+1}\epsilon l^{\alpha}L(nl)\notag\\
			&\gtrsim 2\epsilon n^{\alpha+1}L(n)l^{\alpha+1}.\notag
		\end{align}
		Follows
		\[
		l\lesssim (\frac{1}{2\epsilon(\alpha+1)})^{\frac{1}{\alpha+1}}.
		\]
		Now $l$ and $k$ are bounded above and below, so  we can use uniform convergence theorem (see Theorem 1.2.1 in \cite{bingham1987regular})
		\begin{align}
			\sum_{i=nl}^{nk} f(i) &\sim \sum_{i=nl}^{nk} (i)^\alpha L(n)\notag\\
			&=L(n)\sum_{i=nl}^{nk} i^\alpha\notag\\
			&\sim L(n)\int_{nl}^{nk} t^\alpha dt\notag\\
			&=\frac{1}{\alpha+1}L(n)n^{\alpha+1}(k^{\alpha+1}-l^{\alpha+1}).\notag
		\end{align}
	\end{proof}
	
	\begin{Lemma}
		Let $\epsilon>0$ and $i>\epsilon n$, then $s_i(x)$ has the following asymptotic
		\[
		s_i(x)\sim n\phi^{-1}(\frac{i}{n})
		\]
		where $\phi(x)=(1+x^{\alpha+1})^{\frac{1}{\alpha+1}}-x$.
	\end{Lemma}
	
	\begin{proof}
		Let $l,k>0$ be such that $n(k-l)=i$ and
		\[
		\sum_{i=nl}^{nk} f(i)\le x<\sum_{i=nl}^{nk+1} f(i)
		\]
		Clearly it follows that $l$ and $k$ are unique. Using Lemma~\ref{sum f range asymp} and Lemma~\ref{sum f Asymp}
		\[
		\frac{1}{\alpha+1}L(n)n^{\alpha+1}(k^{\alpha+1}-l^{\alpha+1})\sim \sum_{i=nl}^{nk} f(i)\sim x\sim \sum_{i=1}^{n} f(i) \sim \frac{n^{\alpha+1}L(n)}{\alpha+1}.
		\]
		Follows
		\begin{align}
		\frac{1}{\alpha+1}L(n)n^{\alpha+1}(k^{\alpha+1}-l^{\alpha+1})&\sim \frac{n^{\alpha+1}L(n)}{\alpha+1}\notag\\
		k^{\alpha+1}-l^{\alpha+1}&\sim 1\notag\\
		k-l&\sim (1+l^{\alpha+1})^{\frac{1}{\alpha+1}}-l\notag\\
		i&\sim n((1+l^{\alpha+1})^{\frac{1}{\alpha+1}}-l)\notag\\
		ln&\sim n\phi^{-1}(\frac{i}{n})\notag
		\end{align}
		Since $s_i(x)$ counts the number of blocks whose sum is less that $x$ of length $i$, we get
		\[
		s_i(x)\sim nl\sim n\phi^{-1}(\frac{i}{n})
		\]
	\end{proof}
	
	\begin{Lemma}
		The function $\phi:(0,\infty)\to(0,1)$, $\phi(x)=(1+x^{\alpha+1})^{\frac{1}{\alpha+1}}-x$ is strictly decreasing bijection hence has inverse and
		\[
		\int_{0}^{1} \phi^{-1}(t)dt=\frac{1}{2(\alpha+1)}
		\frac{\Gamma(\frac{1}{\alpha+1})\Gamma(\frac{\alpha-1}{\alpha+1})}{\Gamma(\frac{\alpha}{\alpha+1})}
		\]
	\end{Lemma}
	
	\begin{proof}
		Taking the derivative of $\phi$
		\begin{align}
			\diff{\phi}{x}&=(1+x^{\alpha+1})^{-\frac{\alpha}{\alpha+1}}x^{\alpha}-1\notag\\
			&=\frac{x^\alpha}{((1+x^{\alpha+1})^{\frac{1}{\alpha+1}})^{\alpha}}-1\notag\\
			&<1-1=0
		\end{align}
		Where last inequality follows from $(1+x^{\alpha+1})^{\frac{1}{\alpha+1}}>x$. Because $\phi$'s derivative is negative everywhere, $\phi$ is strictly decreasing. Now $\phi$ is bijection, because $\phi(0)=1$ and $\lim_{x\to\infty}\phi(x)=0$, hence has inverse.
		
		Since $\phi$ is strictly decreasing bijection
		\[
		\int_{0}^{1} \phi^{-1}(x)dx=\int_{0}^{\infty} \phi(x)dx.
		\]
		Doing change of variables $t=\frac{x^{\alpha+1}}{1+x^{\alpha+1}}$, $x=(\frac{t}{1-t})^{\frac{1}{\alpha+1}}$. We get
		\[
		dx=\frac{1}{\alpha+1}\big(\frac{t}{1-t}\big)^{-\frac{\alpha}{\alpha+1}}(1-t)^{-2}dt
		\]
		Changing $\phi(x)$ to $t$ terms \\
		\begin{align}
			\phi(x)&=(1+x^{\alpha+1})^{\frac{1}{\alpha+1}}-x \notag \\
			&=(\frac{1}{1-t})^{\frac{1}{\alpha+1}}-(\frac{t}{1-t})^{\frac{1}{\alpha+1}} \notag	\\
			&=(1-t)^{-\frac{1}{\alpha+1}}(1-t^{\frac{1}{\alpha+1}})\notag
		\end{align}
		Putting it together
		\begin{align}
			\int_{0}^{\infty} \phi(x)dx&=\int_{0}^{1} (1-t)^{-\frac{1}{\alpha+1}}(1-t^{\frac{1}{\alpha+1}})
			\frac{1}{\alpha+1}\big(\frac{t}{1-t}\big)^{-\frac{\alpha}{\alpha+1}}(1-t)^{-2}dt\notag\\
			&=\frac{1}{\alpha+1}\int_{0}^{1} (1-t)^{\frac{\alpha}{\alpha+1}-\frac{1}{\alpha+1}-2}
			(t^{-\frac{\alpha}{\alpha+1}}-t^{\frac{1}{\alpha+1}-\frac{\alpha}{\alpha+1}})dt\notag\\
			&=\frac{1}{\alpha+1}\int_{0}^{1} (1-t)^{-\frac{\alpha+3}{\alpha+1}}
			(t^{-\frac{\alpha}{\alpha+1}}-t^{\frac{1-\alpha}{\alpha+1}})dt\notag\\
			&=\frac{1}{\alpha+1}\int_{0}^{1} (1-t)^{-\frac{\alpha+3}{\alpha+1}}t^{-\frac{\alpha}{\alpha+1}}dt-\frac{1}{\alpha+1}\int_{0}^{1} (1-t)^{-\frac{\alpha+3}{\alpha+1}}t^{\frac{1-\alpha}{\alpha+1}}dt\notag\\
			&=\frac{1}{\alpha+1}\lim_{\epsilon\to 0^+}\int_{0}^{1} (1-t)^{\epsilon-\frac{\alpha+3}{\alpha+1}}t^{-\frac{\alpha}{\alpha+1}}dt-\frac{1}{\alpha+1}\int_{0}^{1} (1-t)^{\epsilon-\frac{\alpha+3}{\alpha+1}}t^{\frac{1-\alpha}{\alpha+1}}dt\notag\\
			&=\frac{1}{\alpha+1}\lim_{\epsilon\to 0^+}\big(B(\frac{1}{\alpha+1}, \epsilon-\frac{2}{\alpha+1})-
			B(\frac{2}{\alpha+1}, \epsilon-\frac{2}{\alpha+1})\big)\notag
		\end{align}
		Where $B$ is the analytically extended beta function(see NIST DLMF § 5.12 \cite{NIST:DLMF})
		\[
		B(z_1, z_2)=\int_{0}^{1} t^{z_1-1}(1-t)^{z_2-1}dt=\frac{\Gamma(z_1)\Gamma(z_2)}{\Gamma(z_1+z_2)}
		\]
		Using gamma expression for beta functions we get
		\begin{align}
			\int_{0}^{\infty} \phi(x)dx&=\frac{1}{\alpha+1}\lim_{\epsilon\to 0^+}(
			\frac{\Gamma(\frac{1}{\alpha+1})\Gamma(\epsilon-\frac{2}{\alpha+1})}{\Gamma(\frac{1}{\alpha+1}+\epsilon-\frac{2}{\alpha+1})}
			-
			\frac{\Gamma(\frac{2}{\alpha+1})\Gamma(\epsilon-\frac{2}{\alpha+1})}{\Gamma(\frac{2}{\alpha+1}+\epsilon-\frac{2}{\alpha+1})}
			)\notag\\
			&=\frac{1}{\alpha+1}\lim_{\epsilon\to 0^+}(
			\frac{\Gamma(\frac{1}{\alpha+1})\Gamma(-\frac{2}{\alpha+1})}{\Gamma(-\frac{1}{\alpha+1})}
			-
			\frac{\Gamma(\frac{2}{\alpha+1})\Gamma(-\frac{2}{\alpha+1})}{\Gamma(\epsilon)}
			)\notag
		\end{align}
		Since $\Gamma(t)=\frac{\Gamma(1+t)}{t}$ we have $\Gamma(t)\sim\frac{1}{t}, t\to 0$. Follows
		\[
		\lim_{\epsilon\to 0^+}\frac{\Gamma(\frac{2}{\alpha+1})\Gamma(-\frac{2}{\alpha+1})}{\Gamma(\epsilon)}=0.
		\]
		So the expression simplifies to
		\begin{align}
			\int_{0}^{\infty} \phi(x)dx&=\frac{1}{\alpha+1}
			\frac{\Gamma(\frac{1}{\alpha+1})\Gamma(-\frac{2}{\alpha+1})}{\Gamma(-\frac{1}{\alpha+1})}
			\notag\\
			&=\frac{1}{2(\alpha+1)}
			\frac{\Gamma(\frac{1}{\alpha+1})\Gamma(\frac{\alpha-1}{\alpha+1})}{\Gamma(\frac{\alpha}{\alpha+1})}
			\notag
		\end{align}
		Where last equality uses $z\Gamma(z)=\Gamma(z+1)$.
	\end{proof}
	
	\begin{Theorem}
		The function $S(x)$ has asymptotic $C(\alpha)n^2$, where $C(\alpha)=\frac{1}{2(\alpha+1)}
		\frac{\Gamma(\frac{1}{\alpha+1})\Gamma(\frac{\alpha-1}{\alpha+1})}{\Gamma(\frac{\alpha}{\alpha+1})}$.
	\end{Theorem}
	
	\begin{proof}
		From definition of $S$ and $s_i$ it follows that
		\[
		S(x)=\sum_{i=1}^{n} s_i(x)
		\]
		%needs lemma on neglibility of head of sum.
	\end{proof}
	
	\bibliographystyle{plain}
	\bibliography{references}
\end{document}