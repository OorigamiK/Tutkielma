\documentclass{article}
\usepackage{amsmath, amssymb, amsthm, url, hyperref}
\usepackage[T1]{fontenc}
\usepackage[utf8]{inputenc}
\usepackage{cleveref}
\usepackage[thinc]{esdiff}
\usepackage{commath}

\newtheorem{lemma}{Lemma}
\newtheorem{theorem}{Theorem}

\let\epsilon\varepsilon

\begin{document}
	\section{Introduction}
	
	\section{Notation}
	\begin{itemize}
		\item Let $RV_\alpha^+$ denote eventually monotonic regularly varying functions bounded away from $0$ and $\infty$ in any closed interval with index $\alpha$ (see \cite{bingham1987regular}). We assume the domain and range are $\mathbb{R}_{>0}$
		\item Let $S(x)=\#\{(i, j)\mid i\le j, f(i)+f(i+1)+\dots+f(j)\le x\}$
		\item Let $s_k(x)=\#\{(i, j)\mid j-i=k, f(i)+f(i+1)+\dots+f(j)\le x\}$
		\item Let $\sim$ denote asymptotic equivalence. That is, 
		\[
		f(x)\sim g(x), x\to a\iff \lim_{x\to a} \frac{f(x)}{g(x)} = 1.
		\]
		We often omit the "$x\to a$" part, when it is clear from context.
		\item Denote $f(x)\lesssim g(x) \iff \limsup\frac{f(x)}{g(x)}\le1$ and similarly 
		\[f(x)\gtrsim g(x) \iff \liminf\frac{f(x)}{g(x)}\ge1\].
	\end{itemize}
	\section{Preliminaries}
	We assume throughout this paper that $n=n(x)$ is the unique integer satisfying
	\[
	\sum_{i=1}^{n} f(i)\le x<\sum_{i=1}^{n+1} f(i).
	\]
	Where $f\in RV_\alpha^+$.
	
	\begin{lemma}\label{sum f Asymp}
		We have
		\[
		x\sim \frac{n^{\alpha+1}L(n)}{\alpha+1}
		\]
	\end{lemma}
	\begin{proof}
		Let $\epsilon>0$ and split the sum
		\[
		\sum_{i=1}^{n} f(i)=\sum_{i=1}^{n\epsilon} f(i)+\sum_{i=n\epsilon}^{n} f(i).
		\]
		We will first examine the second sum.
		\[
		\sum_{i=n\epsilon}^{n} f(i)=\sum_{i=n\epsilon}^{n} i^\alpha L(i)\notag\\
		\]
		Using uniform convergence theorem (see Theorem 1.2.1 in \cite{bingham1987regular})
		\begin{align}
			\sum_{i=n\epsilon}^{n} i^\alpha L(i)&\sim L(n)\sum_{i=n\epsilon}^{n} i^\alpha\notag\\
			&\sim L(n)\int_{n\epsilon}^{n} t^\alpha dt\notag\\
			&=\frac{1}{\alpha+1}L(n)n^{\alpha+1}(1-\epsilon^{\alpha+1})\notag
		\end{align}
		Examining the first sum
		\[
		0<\sum_{i=1}^{n\epsilon} f(i)<n\epsilon f(n\epsilon).
		\]
		Since $f$ is eventually increasing
		\[
		n\epsilon f(n\epsilon)< n\epsilon f(n)=\epsilon n^{\alpha+1}L(n).
		\]
		Combining these we get
		\begin{align}
			\frac{1}{\alpha+1}n^{\alpha+1}L(n)(1-\epsilon^{\alpha+1})\lesssim\sum_{i=1}^{n} f(i)&\lesssim \frac{1}{\alpha+1}n^{\alpha+1}L(n)(1+\epsilon(\alpha+1)-\epsilon^{\alpha+1})\notag\\
			\sum_{i=1}^{n} f(i)&\sim \frac{1}{\alpha+1}n^{\alpha+1}L(n)\notag
		\end{align}
	\end{proof}
	
	\begin{lemma}\label{sum f range asymp}
		Let $\epsilon>0$ given $k,l>0$ such that $k-l>\epsilon$. If \[\sum_{i=nl}^{nk} f(i)<x\] then
		\[
		\sum_{i=nl}^{nk} f(i)\sim \frac{1}{\alpha+1}L(n)n^{\alpha+1}(k^{\alpha+1}-l^{\alpha+1}).
		\]
	\end{lemma}
	\begin{proof}
		From definition of $n$ it follows $k\le l+1$. Since $f$ is eventually increasing
		\[
		\sum_{i=nl}^{nk} f(i)>n(k-l) f(nl)>n\epsilon f(nl).
		\]
		Using Potter's bound (see Theorem 1.5.6 \cite{bingham1987regular}) and Lemma~\ref{sum f Asymp} %laita potter ja lemma 1 eri kohtiin
		\begin{align}
			\frac{1}{\alpha+1}n^{\alpha+1}L(n)&\gtrsim n\epsilon f(nl)\notag\\
			&= n^{\alpha+1}\epsilon l^{\alpha}L(nl)\notag\\
			&\gtrsim \epsilon n^{\alpha+1}L(n)l^{\alpha}\frac{1}{2\max(l, l^{-1})}.\notag
		\end{align}
		Follows
		\[
		l\lesssim (\frac{1}{2\epsilon(\alpha+1)})^{\frac{1}{\alpha+1}}.
		\]
		Now $l$ and $k$ are bounded above and below, so  we can use uniform convergence theorem (see Theorem 1.2.1 in \cite{bingham1987regular})
		\begin{align}
			\sum_{i=nl}^{nk} f(i) &\sim \sum_{i=nl}^{nk} i^\alpha L(n)\notag\\
			&=L(n)\sum_{i=nl}^{nk} i^\alpha\notag\\
			&\sim L(n)\int_{nl}^{nk} t^\alpha dt\notag\\
			&=\frac{1}{\alpha+1}L(n)n^{\alpha+1}(k^{\alpha+1}-l^{\alpha+1}).\notag
		\end{align}
	\end{proof}
	
	\begin{lemma}\label{s asymp}
		Let $\epsilon>0$ and $i>\epsilon n$, then $s_i(x)$ has the following asymptotic
		\[
		s_i(x)\sim n\phi^{-1}\left(\frac{i}{n}\right)
		\]
		where $\phi(x)=(1+x^{\alpha+1})^{\frac{1}{\alpha+1}}-x$.
	\end{lemma}
	
	\begin{proof}
		Let $l,k>0$ be such that $n(k-l)=i$ and
		\[
		\sum_{j=nl}^{nk} f(j)\le x<\sum_{j=nl}^{nk+1} f(j)
		\]
		Clearly it follows that $l$ and $k$ are unique. Using Lemma~\ref{sum f range asymp} and Lemma~\ref{sum f Asymp}
		\[
		\frac{1}{\alpha+1}L(n)n^{\alpha+1}(k^{\alpha+1}-l^{\alpha+1})\sim \sum_{j=nl}^{nk} f(j)\sim x\sim \sum_{j=1}^{n} f(j) \sim \frac{n^{\alpha+1}L(n)}{\alpha+1}.
		\]
		Follows
		\begin{align}
			\frac{n^{\alpha+1}L(n)}{\alpha+1}(k^{\alpha+1}-l^{\alpha+1})&\sim \frac{n^{\alpha+1}L(n)}{\alpha+1}\notag\\
			k^{\alpha+1}-l^{\alpha+1}&\sim 1\notag\\
			k-l&\sim (1+l^{\alpha+1})^{\frac{1}{\alpha+1}}-l\notag\\
			i&\sim n((1+l^{\alpha+1})^{\frac{1}{\alpha+1}}-l)\notag\\
			ln&\sim n\phi^{-1}(\frac{i}{n})\notag
		\end{align}
		Since $s_i(x)$ counts the number of blocks whose sum is less than $x$ of length $i$, we get
		\[
		s_i(x) = nl\sim n\phi^{-1}(\frac{i}{n}).
		\]
	\end{proof}
	\section{Case $\alpha=1$}

	
	
	For $\alpha=1$ the crude bounds work. Clearly
	\[
	\sum_{i=1}^{n} f^{-1}(\frac{x}{i})-i<S(x)<\sum_{i=1}^{n} f^{-1}(\frac{x}{i}).
	\]
	Doing change of variables we get
	\begin{align}
	\sum_{i=1}^{n} f^{-1}(\frac{x}{i})&\sim x \int_{1}^{n} t^{-1}{L}(\frac{x}{t})dt \notag\\
	&=x\int_{x/n}^{x} \frac{\tilde{L}(u)}{u} du \notag\\
	&>x\int_{1}^{x} \frac{\tilde{L}(u)}{u} du\notag
	\end{align}
	From Karamata's Theorem (direct half) (see proposition 1.5.8~\cite{bingham1987regular}) it follows
	\[
	\int_{1}^{x} \frac{\tilde{L}(u)}{u} du>M\tilde{L}(x),
	\]
	given large enough $x$. 
	\section{Case $\alpha>1$}
	\begin{lemma}\label{phi integral}
		The function $\phi:(0,\infty)\to(0,1)$, $\phi(x)=(1+x^{\alpha+1})^{\frac{1}{\alpha+1}}-x$ is strictly decreasing bijection hence has inverse and
		\[
		\int_{0}^{1} \phi^{-1}(t)dt=\frac{1}{2(\alpha+1)}
		\frac{\Gamma(\frac{1}{\alpha+1})\Gamma(\frac{\alpha-1}{\alpha+1})}{\Gamma(\frac{\alpha}{\alpha+1})}
		\]
	\end{lemma}
	
	\begin{proof}
		Taking the derivative of $\phi$
		\begin{align}
			\diff{\phi}{x}&=(1+x^{\alpha+1})^{-\frac{\alpha}{\alpha+1}}x^{\alpha}-1\notag\\
			&=\frac{x^\alpha}{((1+x^{\alpha+1})^{\frac{1}{\alpha+1}})^{\alpha}}-1\notag\\
			&<1-1=0\notag
		\end{align}
		Where last inequality follows from $(1+x^{\alpha+1})^{\frac{1}{\alpha+1}}>x$. Because $\phi$'s derivative is negative everywhere, $\phi$ is strictly decreasing. Now $\phi$ is bijection, because $\phi(0)=1$ and $\lim_{x\to\infty}\phi(x)=0$, hence has inverse.
		
		Since $\phi$ is strictly decreasing bijection
		\[
		\int_{0}^{1} \phi^{-1}(x)dx=\int_{0}^{\infty} \phi(x)dx.
		\]
		Next we show that the integral converges. Since $x^{\frac{1}{\alpha+1}}$ is increasing with decreasing derivative, we have $(1+y)^{\frac{1}{\alpha+1}}<y^{\frac{1}{\alpha+1}}+\frac{1}{\alpha+1}y^{-\frac{\alpha}{\alpha+1}}$. Using this we get
		\[
		\phi(x)<x^{-\alpha},
		\]
		so the integral converges.
		
		Doing change of variables $t=\frac{x^{\alpha+1}}{1+x^{\alpha+1}}$, $x=(\frac{t}{1-t})^{\frac{1}{\alpha+1}}$. We get
		\[
		dx=\frac{1}{\alpha+1}\big(\frac{t}{1-t}\big)^{-\frac{\alpha}{\alpha+1}}(1-t)^{-2}dt.
		\]
		Changing $\phi(x)$ to $t$ terms \\
		\begin{align}
			\phi(x)&=(1+x^{\alpha+1})^{\frac{1}{\alpha+1}}-x \notag \\
			&=(\frac{1}{1-t})^{\frac{1}{\alpha+1}}-(\frac{t}{1-t})^{\frac{1}{\alpha+1}} \notag	\\
			&=(1-t)^{-\frac{1}{\alpha+1}}(1-t^{\frac{1}{\alpha+1}}).\notag
		\end{align}
		Putting it together
		\begin{align}
			\int_{0}^{\infty} \phi(x)dx&=\int_{0}^{1} (1-t)^{-\frac{1}{\alpha+1}}(1-t^{\frac{1}{\alpha+1}})
			\frac{1}{\alpha+1}\big(\frac{t}{1-t}\big)^{-\frac{\alpha}{\alpha+1}}(1-t)^{-2}dt\notag\\
			&=\frac{1}{\alpha+1}\int_{0}^{1} (1-t)^{\frac{\alpha}{\alpha+1}-\frac{1}{\alpha+1}-2}
			(t^{-\frac{\alpha}{\alpha+1}}-t^{\frac{1}{\alpha+1}-\frac{\alpha}{\alpha+1}})dt\notag\\
			&=\frac{1}{\alpha+1}\int_{0}^{1} (1-t)^{-\frac{2}{\alpha+1}-1}
			(t^{-\frac{\alpha}{\alpha+1}}-t^{\frac{1-\alpha}{\alpha+1}})dt\notag
		\end{align}
		
		
		
		Since $\lim_{r\to 0} (1-t)^r=1, t\in (0,1)$, we can take any $\epsilon, \delta>0$ such that $\abs{(1-t)^r-1}<\epsilon, t\in[\delta, 1-\delta]$ for small enough $r>0$.
		Since the integral converges
		\begin{align}
			\int_{0}^{1-\delta} (1-t)^{-\frac{2}{\alpha+1}-1}
			(t^{-\frac{\alpha}{\alpha+1}}-t^{\frac{1-\alpha}{\alpha+1}})dt\notag\\
			=\int_{0}^{1} (1-t)^{-\frac{2}{\alpha+1}-1}
			(t^{-\frac{\alpha}{\alpha+1}}-t^{\frac{1-\alpha}{\alpha+1}})dt+h(\delta)\notag
		\end{align}
		Where $\lim_{\delta\to 0} h(\delta)=0$.
		\begin{align}
			&\abs{\int_{0}^{1-\delta} (1-t)^{r-\frac{2}{\alpha+1}-1}
			(t^{-\frac{\alpha}{\alpha+1}}-t^{\frac{1-\alpha}{\alpha+1}})dt-
			\int_{0}^{1} (1-t)^{-\frac{2}{\alpha+1}-1}
			(t^{-\frac{\alpha}{\alpha+1}}-t^{\frac{1-\alpha}{\alpha+1}})dt}\notag\\
			&=\abs{\int_{0}^{1-\delta} (1-t)^{r-\frac{2}{\alpha+1}-1}
			(t^{-\frac{\alpha}{\alpha+1}}-t^{\frac{1-\alpha}{\alpha+1}})dt-
			\int_{0}^{1-\delta} (1-t)^{-\frac{2}{\alpha+1}-1}
			(t^{-\frac{\alpha}{\alpha+1}}-t^{\frac{1-\alpha}{\alpha+1}})dt+h(\delta)}\notag\\
			&=\abs{\int_{0}^{1-\delta} ((1-t)^{r}-1)(1-t)^{-\frac{2}{\alpha+1}-1}
			(t^{-\frac{\alpha}{\alpha+1}}-t^{\frac{1-\alpha}{\alpha+1}})dt+h(\delta)}\notag\\
			&<\epsilon \int_{0}^{1-\delta} (1-t)^{-\frac{2}{\alpha+1}-1}
			(t^{-\frac{\alpha}{\alpha+1}}-t^{\frac{1-\alpha}{\alpha+1}})dt+\abs{h(\delta)}\notag
		\end{align}
		Letting $\epsilon, \delta\to 0$, gives that the limit can be taken out of the integral
		\begin{align}
			\int_{0}^{1} (1-t)^{-\frac{2}{\alpha+1}-1}
			(t^{-\frac{\alpha}{\alpha+1}}-t^{\frac{1-\alpha}{\alpha+1}})dt\notag\\
			=\lim_{r\to0} \int_{0}^{1} (1-t)^{r-\frac{2}{\alpha+1}-1}
			(t^{-\frac{\alpha}{\alpha+1}}-t^{\frac{1-\alpha}{\alpha+1}})dt\notag
		\end{align}
		
		Let $B$ be the beta function (see NIST DLMF § 5.12.1 \cite{NIST:DLMF})
		\[
		B(z_1, z_2)=\int_{0}^{1} t^{z_1-1}(1-t)^{z_2-1}dt=\frac{\Gamma(z_1)\Gamma(z_2)}{\Gamma(z_1+z_2)}
		\]
		The beta function can be analytically extended to $\left(\mathbb{C}\setminus \mathbb{Z},\mathbb{C}\setminus \mathbb{Z}\right)$ using Pochhammer's integral (see NIST DLMF § 5.12.12).

		
		
		\begin{align}
			&=\frac{1}{\alpha+1}\int_{0}^{1} (1-t)^{-\frac{2}{\alpha+1}-1}
			(t^{-\frac{\alpha}{\alpha+1}}-t^{\frac{1-\alpha}{\alpha+1}})dt\notag\\
			&=\frac{1}{\alpha+1} \lim_{r\to0} \int_{0}^{1} (1-t)^{r-\frac{2}{\alpha+1}-1}
			(t^{-\frac{\alpha}{\alpha+1}}-t^{\frac{1-\alpha}{\alpha+1}})dt\notag\\
			&=\frac{1}{\alpha+1}\lim_{r\to 0^+}\left(B(\frac{1}{\alpha+1}, r-\frac{2}{\alpha+1})-
			B(\frac{2}{\alpha+1}, r-\frac{2}{\alpha+1})\right).\notag
		\end{align}
		Using gamma expression for beta functions we get
		\begin{align}
			\int_{0}^{\infty} \phi(x)dx&=\frac{1}{\alpha+1}\lim_{r\to 0}\left(
			\frac{\Gamma(\frac{1}{\alpha+1})\Gamma(r-\frac{2}{\alpha+1})}{\Gamma(\frac{1}{\alpha+1}+r-\frac{2}{\alpha+1})}
			-
			\frac{\Gamma(\frac{2}{\alpha+1})\Gamma(r-\frac{2}{\alpha+1})}{\Gamma(\frac{2}{\alpha+1}+r-\frac{2}{\alpha+1})}
			\right)\notag\\
			&=\frac{1}{\alpha+1}\lim_{r\to 0}\left(
			\frac{\Gamma(\frac{1}{\alpha+1})\Gamma(-\frac{2}{\alpha+1})}{\Gamma(-\frac{1}{\alpha+1})}
			-
			\frac{\Gamma(\frac{2}{\alpha+1})\Gamma(-\frac{2}{\alpha+1})}{\Gamma(r)}
			\right).\notag
		\end{align}
		Since $\Gamma(t)=\frac{\Gamma(1+t)}{t}$ we have $\Gamma(t)\sim\frac{1}{t}, t\to 0$. Follows
		\[
		\lim_{r\to 0}\frac{\Gamma(\frac{2}{\alpha+1})\Gamma(-\frac{2}{\alpha+1})}{\Gamma(r)}=0.
		\]
		So the expression simplifies to
		\begin{align}
			\int_{0}^{\infty} \phi(x)dx&=\frac{1}{\alpha+1}
			\frac{\Gamma(\frac{1}{\alpha+1})\Gamma(-\frac{2}{\alpha+1})}{\Gamma(-\frac{1}{\alpha+1})}
			\notag\\
			&=\frac{1}{2(\alpha+1)}
			\frac{\Gamma(\frac{1}{\alpha+1})\Gamma(\frac{\alpha-1}{\alpha+1})}{\Gamma(\frac{\alpha}{\alpha+1})},
			\notag
		\end{align}
		where last equality uses $z\Gamma(z)=\Gamma(z+1)$.
	\end{proof}
	
	\begin{lemma}\label{head's neglible}
		Let $\epsilon>0$ then 
		\[
		\sum_{i=1}^{n\epsilon} s_i(x)=O(n^2)\epsilon^{\frac{1}{2}(1-\frac{1}{\alpha})}.
		\]
	\end{lemma}
	
	\begin{proof}
		Clearly $s_i(x)\le f^{-1}(\frac{x}{i})$. Regularly varying functions inverse is regularly varying with index $\frac{1}{\alpha}$ (see Theorem 1.5.12 in \cite{bingham1987regular}). Let $f^{-1}(x)=x^{\frac{1}{\alpha}}\tilde{L}(x)$.
		\[
		\sum_{i=1}^{n\epsilon} f^{-1}(\frac{x}{i})\sim  \int_{1}^{n\epsilon} f^{-1}(\frac{x}{t})dt=x^{\frac{1}{\alpha}}\int_{1}^{n\epsilon} t^{-\frac{1}{\alpha}}\tilde{L}(\frac{x}{t})dt
		\]
		By Potter's bound
		\[
		\tilde{L}(\frac{x}{t})\le M\tilde{L}(\frac{x}{n})(\frac{n}{t})^{\delta},
		\]
		where $M$ is the constant from Potter's bound and $\delta=\frac{1}{2}(1-\frac{1}{\alpha})$. Now we can bound the integral
		\begin{align}
			x^{\frac{1}{\alpha}}\int_{1}^{n\epsilon} t^{-\frac{1}{\alpha}}\tilde{L}(\frac{x}{t})dt&\le
			x^{\frac{1}{\alpha}}M\tilde{L}(\frac{x}{n})n^\delta\int_{1}^{n\epsilon} t^{-\frac{1}{\alpha}-\delta}\notag\\
		\end{align}
		We know that
		\begin{align}
			f^{-1}(f(n))=n\notag\\
			n\tilde{L}(n^\alpha L(n))L(n)^{1/\alpha}\sim n\notag\\
			\tilde{L}(n^\alpha L(n))L(n)^{1/\alpha}\sim 1\notag
		\end{align}
		Follows
		\begin{align}
			x^{\frac{1}{\alpha}}M\tilde{L}(\frac{x}{n})n^\delta\int_{1}^{n\epsilon} t^{-\frac{1}{\alpha}-\delta}
			&\sim Mn^{2}L(n)^{\frac{1}{\alpha}}\tilde{L}(n^\alpha L(n))\epsilon^{\frac{1}{2}(1-\frac{1}{\alpha})}\notag\\
			&=O(n^2)\epsilon^{\frac{1}{2}(1-\frac{1}{\alpha})}\notag
		\end{align}		
	\end{proof}
	
	\begin{theorem}
		The function $S(x)$ has asymptotic $C(\alpha)n^2$, where 
		\[
		C(\alpha)=\frac{1}{2(\alpha+1)}
		\frac{\Gamma(\frac{1}{\alpha+1})\Gamma(\frac{\alpha-1}{\alpha+1})}{\Gamma(\frac{\alpha}{\alpha+1})}.
		\]
	\end{theorem}
	
	\begin{proof}
		Let $\epsilon>0$. From definition of $S$ and $s_i$ it follows that
		\[
		S(x)=\sum_{i=1}^{n} s_i(x)
		\]
		Splitting the sum
		\[
		S(x)=\sum_{i=1}^{n\epsilon} s_i(x)+\sum_{i=n\epsilon}^{n} s_i(x)
		\]
		Using lemma~\ref{head's neglible} we can control the first sum, so we will focus on the second sum. Using lemma~\ref{s asymp} we get
		\begin{align}
			\sum_{i=n\epsilon}^{n} s_i(x)&\sim \sum_{i=n\epsilon}^{n} n\phi^{-1}(\frac{i}{n})\notag\\
			&=n\sum_{i=n\epsilon}^{n} \phi^{-1}(\frac{i}{n})\notag\\
			&\sim n \int_{n\epsilon}^{n} \phi^{-1}(\frac{t}{n})dt\notag\\
			&=n^2\int_{\epsilon}^{1} \phi^{-1}(t)dt.\notag
		\end{align}
		We can bound $\phi^{-1}$, using the bound derived in Lemma~\ref{phi integral}
		\begin{align}
			\phi(x)&<x^{-\alpha}\notag\\
			\phi(x^{-\frac{1}{\alpha}})&<x\notag\\
			x^{-\frac{1}{\alpha}}&<\phi^{-1}(x)\notag
		\end{align}
		Now we can bound
		\begin{align}
			\int_{0}^{\epsilon} \phi^{-1}(t)dt&=\epsilon\phi^{-1}(\epsilon)+\int_{\phi^{-1}(\epsilon)}^{\infty} \phi(t)dt\notag\\
			&<\epsilon^{1-\frac{1}{\alpha}}+\phi^{-1}(\epsilon)^{1-\alpha}.\notag
		\end{align}
		We get
		\begin{align}
			n^2\int_{\epsilon}^{1} \phi^{-1}(t)dt=n^2(\int_{0}^{1} \phi^{-1}(t)dt-\int_{0}^{\epsilon} \phi^{-1}(t)dt)\notag\\
			>n^2(C(\alpha)-\epsilon^{1-\frac{1}{\alpha}}+\phi^{-1}(\epsilon)^{1-\alpha})\notag
		\end{align}
		and
		\begin{align}
			n^2\int_{\epsilon}^{1} \phi^{-1}(t)dt<n^2C(\alpha).\notag
		\end{align}
		Combining this and bound from Lemma~\ref{head's neglible} the Theorem follows.
	\end{proof}
	
	Assume \(\alpha>1\). We start from
	\[
	I:=\int_0^\infty \phi(x)\,dx
	=\int_0^\infty\Big((1+x^{\alpha+1})^{\frac{1}{\alpha+1}}-x\Big)\,dx.
	\]
	
	Make the substitution
	\[
	t=\frac{x^{\alpha+1}}{1+x^{\alpha+1}},\qquad t\in(0,1).
	\]
	Equivalently set \(s=t^{\frac{1}{\alpha+1}}\in(0,1)\). We record useful identities.
	Since \(t=\dfrac{x^{\alpha+1}}{1+x^{\alpha+1}}\) we have
	\[
	1-t=\frac{1}{1+x^{\alpha+1}},\qquad
	(1+x^{\alpha+1})^{\frac{1}{\alpha+1}}=(1-t)^{-\frac{1}{\alpha+1}},
	\]
	and
	\[
	x=\Big(\frac{t}{1-t}\Big)^{\frac{1}{\alpha+1}}.
	\]
	Writing \(t=s^{\alpha+1}\) gives the compact expressions
	\[
	x=s\,(1-s^{\alpha+1})^{-\frac{1}{\alpha+1}},\qquad
	(1+x^{\alpha+1})^{\frac{1}{\alpha+1}}=(1-s^{\alpha+1})^{-\frac{1}{\alpha+1}}.
	\]
	
	Hence
	\[
	\phi(x)=(1+x^{\alpha+1})^{\frac{1}{\alpha+1}}-x
	=(1-s^{\alpha+1})^{-\frac{1}{\alpha+1}}(1-s).
	\]
	
	Differentiate \(x(s)=s(1-s^{\alpha+1})^{-\frac{1}{\alpha+1}}\). A short calculation yields
	\[
	\frac{dx}{ds}=(1-s^{\alpha+1})^{-\frac{1}{\alpha+1}-1}.
	\]
	Therefore
	\[
	\phi(x)\,dx
	=(1-s)\,(1-s^{\alpha+1})^{-\frac{1}{\alpha+1}}\cdot(1-s^{\alpha+1})^{-\frac{1}{\alpha+1}-1}\,ds
	=(1-s)\,(1-s^{\alpha+1})^{-\frac{2}{\alpha+1}-1}\,ds.
	\]
	
	Thus the integral becomes
	\[
	I=\int_{0}^{1}(1-s)\,(1-s^{\alpha+1})^{-\frac{2}{\alpha+1}-1}\,ds.
	\]
	
	Now put \(u=s^{\alpha+1}\). Then \(du=(\alpha+1)s^{\alpha}\,ds\), so \(ds=\dfrac{du}{(\alpha+1)u^{\frac{\alpha}{\alpha+1}}}\)
	and \(s=u^{\frac{1}{\alpha+1}}\). Substitute and split the factor \((1-s)=1-u^{\frac{1}{\alpha+1}}\):
	\[
	\begin{aligned}
		I
		&=\frac{1}{\alpha+1}\int_{0}^{1}
		(1-u^{\frac{1}{\alpha+1}})
		\,u^{-\frac{\alpha}{\alpha+1}}\,(1-u)^{-\frac{2}{\alpha+1}-1}\,du\\[6pt]
		&=\frac{1}{\alpha+1}\Bigg[
		\int_{0}^{1} u^{-\frac{\alpha}{\alpha+1}}(1-u)^{-\frac{2}{\alpha+1}-1}\,du
		-
		\int_{0}^{1} u^{\frac{1-\alpha}{\alpha+1}}(1-u)^{-\frac{2}{\alpha+1}-1}\,du
		\Bigg].
	\end{aligned}
	\]
	
	Each integrand is of the Beta form \(u^{a-1}(1-u)^{b-1}\). Identify the parameters:
	\[
	\begin{aligned}
		\text{First integral: }& a_1=1-\frac{\alpha}{\alpha+1}=\frac{1}{\alpha+1},\quad
		b_1=-\frac{2}{\alpha+1};\\[4pt]
		\text{Second integral: }& a_2=\frac{1-\alpha}{\alpha+1}+1=\frac{2}{\alpha+1},\quad
		b_2=-\frac{2}{\alpha+1}.
	\end{aligned}
	\]
	
	We have written the integral as a difference of Beta integrals:
	\[
	I=\frac{1}{\alpha+1}\big(B(a_1,b_1)-B(a_2,b_1)\big),
	\]
	where \(B(a,b)=\int_0^1 u^{a-1}(1-u)^{b-1}\,du\) is the Beta function. The formula \(B(a,b)=\dfrac{\Gamma(a)\Gamma(b)}{\Gamma(a+b)}\)
	extends (by analytic continuation) to these parameter values. For \(\alpha>1\) the combination simplifies without encountering poles:
	\[
	\begin{aligned}
		I
		&=\frac{1}{\alpha+1}\left(
		\frac{\Gamma\!\big(\frac{1}{\alpha+1}\big)\Gamma\!\big(-\frac{2}{\alpha+1}\big)}
		{\Gamma\!\big(-\frac{1}{\alpha+1}\big)}
		-
		\frac{\Gamma\!\big(\frac{2}{\alpha+1}\big)\Gamma\!\big(-\frac{2}{\alpha+1}\big)}
		{\Gamma\!\big(0\big)}
		\right).
	\end{aligned}
	\]
	The second term vanishes in the limit because \(\Gamma(0+r)\sim 1/r\) while the numerator stays finite. Simplifying the remaining Gamma factors by the identity \(z\Gamma(z)=\Gamma(z+1)\) gives
	\[
	I=\frac{1}{2(\alpha+1)}
	\frac{\Gamma\!\big(\frac{1}{\alpha+1}\big)\Gamma\!\big(\frac{\alpha-1}{\alpha+1}\big)}
	{\Gamma\!\big(\frac{\alpha}{\alpha+1}\big)}.
	\]
	
	This equals \(\int_0^1\phi^{-1}(t)\,dt\) because \(\phi\) is a decreasing bijection \((0,\infty)\to(0,1)\).
	
	
	\bibliographystyle{plain}
	\bibliography{references}
\end{document}