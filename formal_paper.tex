\documentclass{article}
\usepackage{amsmath, amssymb, amsthm, url, hyperref}
\usepackage[T1]{fontenc}
\usepackage[utf8]{inputenc}
\usepackage{cleveref}

\newtheorem{Lemma}{Lemma}
\newtheorem{Theorem}{Theorem}

\newcommand{\simlt}{\mathrel{\underset{\sim}{<}}}

\begin{document}
	\section{Notation}
	\begin{itemize}
		\item Let $RV_\alpha^+$ denote eventually monotonic regularly varying functions with index $\alpha$ (see \cite{bingham1987regular}). We assume the domain and range are $\mathbb{R}_{>0}$
		\item Let $S(x)=\#\{(i, j)\mid i\le j, f(i)+f(i+1)+\dots+f(j)\le x\}$
		\item Let $s_k(x)=\#\{(i, j)\mid j-i=k, f(i)+f(i+1)+\dots+f(j)\le x\}$
		\item Let $\sim$ denote asymptotic equivalence. That is, 
		\[
		f(x)\sim g(x), x\to a\iff \lim_{x\to a} \frac{f(x)}{g(x)} = 1.
		\]
		We often omit the "$x\to a$" part, when it is clear from context.
		\item Denote $f(x)\lesssim g(x) \iff \limsup\frac{f(x)}{g(x)}\le1$ and similarly 
		\[f(x)\gtrsim g(x) \iff \limsup\frac{f(x)}{g(x)}\ge1\].
	\end{itemize}
	
	\section{Preliminaries}
	We assume throughout this section that $n=n(x)$ is the unique integer satisfying
	\[
	\sum_{i=1}^{n} f(i)\le x<\sum_{i=1}^{n+1} f(i).
	\]
	Where $f : \mathbb{R}_{>0}\to\mathbb{R}_{>0}$.
	\begin{Lemma}\label{sum f Asymp}
		Given $f\in RV_\alpha^+$ we have
		\[
		x\sim \frac{n^{\alpha+1}L(n)}{\alpha+1}
		\]
	\end{Lemma}
	\begin{proof}
		Let $\epsilon>0$ and split the sum
		\[
		\sum_{i=1}^{n} f(i)=\sum_{i=1}^{n\epsilon} f(i)+\sum_{i=n\epsilon}^{n} f(i).
		\]
		We will first examine the second sum.
		\[
		\sum_{i=n\epsilon}^{n} f(i)=\sum_{i=n\epsilon}^{n} i^\alpha L(i)\notag\\
		\]
		Using uniform convergence theorem (see Theorem 1.2.1 in \cite{bingham1987regular})
		\begin{align}
		\sum_{i=n\epsilon}^{n} i^\alpha L(i)&\sim L(n)\sum_{i=n\epsilon}^{n} i^\alpha\notag\\
		&\sim L(n)\int_{n\epsilon}^{n} t^\alpha dt\notag\\
		&=\frac{1}{\alpha+1}L(n)n^{\alpha+1}(1-\epsilon^{\alpha+1})\notag
		\end{align}
		Examining the first sum
		\[
		0<\sum_{i=1}^{n\epsilon} f(i)<M+n\epsilon f(n\epsilon),
		\]
		for some constant $M$. Since $f$ is eventually increasing
		\[
		M+n\epsilon f(n\epsilon)< M+n\epsilon f(n)=M+\epsilon n^{\alpha+1}L(n).
		\]
		Combining these we get
		\begin{align}
		\frac{1}{\alpha+1}n^{\alpha+1}L(n)(1-\epsilon^{\alpha+1})\lesssim\sum_{i=1}^{n} f(i)&\lesssim M+\frac{1}{\alpha+1}n^{\alpha+1}L(n)(1+\epsilon-\epsilon^{\alpha+1})\notag\\
		\sum_{i=1}^{n} f(i)&\sim \frac{1}{\alpha+1}n^{\alpha+1}L(n)\notag
		\end{align}
	\end{proof}
	
	\begin{Lemma}
		Let $\epsilon>0$ given $k,l>0$ such that $k-l>\epsilon$ and $f\in RV_\alpha^+$. If \[\sum_{i=nl}^{nk} f(i)<x\] then
		\[
		\sum_{i=nl}^{nk} f(i)\sim \frac{1}{\alpha+1}L(n)n^{\alpha+1}(k^{\alpha+1}-l^{\alpha+1}).
		\]
	\end{Lemma}
	\begin{proof}
		From definition of $n$ it follows $k\le l+1$. Since $f$ is eventually increasing
		\[
		\sum_{i=nl}^{nk} f(i)>n(k-l) f(nl)>n\epsilon f(nl).
		\]
		Using Potter's bound (see Theorem 1.5.6 \cite{bingham1987regular}) and Lemma~\ref{sum f Asymp}
		\begin{align}
			\frac{1}{\alpha+1}n^{\alpha+1}L(n)&\gtrsim n\epsilon f(nl)\notag\\
			&= n^{\alpha+1}\epsilon l^{\alpha}L(nl)\notag\\
			&\gtrsim 2\epsilon n^{\alpha+1}L(n)l^{\alpha+1}.\notag
		\end{align}
		Follows
		\[
		l\lesssim (\frac{1}{2\epsilon(\alpha+1)})^{\frac{1}{\alpha+1}}.
		\]
		Now $l$ and $k$ are bounded above and below, so  we can use uniform convergence theorem (see Theorem 1.2.1 in \cite{bingham1987regular})
		\begin{align}
			\sum_{i=nl}^{nk} f(i) &\sim \sum_{i=nl}^{nk} (i)^\alpha L(n)\notag\\
			&=L(n)\sum_{i=nl}^{nk} i^\alpha\notag\\
			&\sim L(n)\int_{nl}^{nk} t^\alpha dt\notag\\
			&=\frac{1}{\alpha+1}L(n)n^{\alpha+1}(k^{\alpha+1}-l^{\alpha+1}).\notag
		\end{align}
	\end{proof}
	
	
	
	
	
	
	\bibliographystyle{plain}
	\bibliography{references}
\end{document}