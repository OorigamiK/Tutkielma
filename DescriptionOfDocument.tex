\documentclass[10pt]{article}

\begin{document}
	- Interval sum means summing elements in a row.\\
	- Asymptotic means looking at the big picture ignoring small bumps.\\
	- Main theorem provides asymptotic for the number of interval sums.\\
	- We first find asymptotic for interval sums of fixed length.\\
	- Then we sum these asymptotic to find the asymptotic for total number of interval sums.\\
	- We find that the interval sums are rare.\\
	- We considered only sequences that have growth similar to polynomial.\\

	
	Take a sequence of numbers, interval sum means summing elements of the sequence in a row. It raises a natural question, how many interval sums are there? Our main theorem provides asymptotic for the number of interval sums, where asymptotic means good approximation for large values. One restriction of our theorem is that it only works for sequences that have growth similar to polynomial.
\end{document} 