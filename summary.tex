\documentclass[10pt]{article}
\usepackage{amsmath, amssymb, amsthm, url, hyperref}
\usepackage[T1]{fontenc}
\usepackage[utf8]{inputenc}
\usepackage{cleveref}
\usepackage[thinc]{esdiff}
\usepackage{commath}
\usepackage{xr-hyper}
\externaldocument{appendix}
\linespread{2}
%\usepackage[margin=25mm]{geometry}
\usepackage{comment}
\usepackage{float}
\usepackage{siunitx}
\usepackage{booktabs}

\title{Asymptotic Behavior of the Number of Interval Sums}
\date{}
\author{}	

\begin{document}
	\textbf{Interval sum} is sum of consecutive terms in a sequence. An asymptotic formula is found for the number of interval sums for a class of sequences $\left(f(i)\right)_{i=1}^{\infty}$, where $f$ is measurable, locally bounded away from 0 and $\infty$, eventually increasing, and regularly varying with index $\alpha> 1$. This generalizes and extends results of O'Sullivan et al. on interval sums of prime powers. 
	
	The main theorem provides an asymptotic formula for the interval sum counting function
	\begin{equation*}
		S(x)=\#\left\{(i, j)\in\mathbb{N}^2\mid i\le j, f(i)+f(i+1)+\dots+f(j)\le x\right\}.
	\end{equation*}
	That is
	\begin{equation*}
		S(x)\sim n^2C(\alpha) \qquad n=n(x)\to\infty
	\end{equation*}
	where $x\sim \frac{nf(n)}{\alpha+1}, x\to\infty$ and 
	\begin{equation*}
		C(\alpha)=\frac{1}{2(\alpha+1)}
		\frac{\Gamma(\frac{1}{\alpha+1})\Gamma(\frac{\alpha-1}{\alpha+1})}{\Gamma(\frac{\alpha}{\alpha+1})}.
	\end{equation*} 
	
	The proof relies on classical results from the theory of regular variation, in particular the uniform convergence theorem and Potter's bound. The function $S(x)$ can be split into a sum of $s_i$ that counts the number of intervals of length $i$. The main idea in the proof is that $s_i$ can be related to the inverse of 
	\begin{equation*}
		\phi(x)=(1+x^{\alpha+1})^{\frac{1}{\alpha+1}}-x
	\end{equation*} 
	which allows us to relate $S(x)$ asymptotically to
	\begin{equation*}
		n^2\int_0^{1} \phi^{-1}(t)dt.
	\end{equation*}
	The integral can be calculated using standard substitution and using the analytically extended beta function, which completes the proof.
	
\end{document}