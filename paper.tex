\documentclass{article}
\usepackage{amsmath, amssymb, amsthm, url, hyperref}  % amssymb for \mathbb, amsthm for lemma
\usepackage[T1]{fontenc}
\usepackage[utf8]{inputenc}
\usepackage{cleveref}

\newtheorem{lemma}{lemma}
\newtheorem{theorem}{theorem}

\begin{document}
	
	\begin{lemma}
		Let $f:\mathbb{R}\to\mathbb{R}$ be a convex, twice differentiable function such that $f'(x) \ge 0$. Let $g:\mathbb{R}\to\mathbb{R}$ satisfy
		\[
		f(g(x)) \sim x
		\]
		Then
		\[
		g(x) \sim f^{-1}(x)
		\]
	\end{lemma}
	
	\begin{proof}
		From $f(g(x))\sim x$ follows
		\[
		f(g(x))=x+o(x)
		\]
		Taking inverse of $f$
		\[
		g(x)=f^{-1}(x+o(x))
		\]
		Since $f$ is convex and monotonic, we get
		\[
		\begin{split}
		f^{-1}((1+\epsilon)x)&\le(1+\epsilon)f^{-1}(x) \\
		f^{-1}((1-\epsilon)x)&\ge(1-\epsilon)f^{-1}(x)
		\end{split}
		\]
		Hence, 
		\[
		g(x)=f^{-1}(x+o(x))=(1+o(1))f^{-1}(x)\sim f^{-1}(x)
		\]
	\end{proof}
	
	\begin{lemma}\label{lem:tail-dominance}
		Let $f$ be rapidly varying function. Fix $\epsilon>0$, then for sufficiently large $n$
		\[
		\sum_{i=1}^{n} f(i)<\sum_{i=n+1}^{n(1+\epsilon)} f(i)
		\]
	\end{lemma}
	
	\begin{proof}
		We will prove that the ratio tends to infinity. We examine the left-hand side
		\[
		\sum_{i=1}^{n}f(i)<nf(n)
		\]
		On the right-hand side
		\[
		\sum_{i=n+1}^{n(1+\epsilon)}f(i)>n\frac{\epsilon}{2} f(n(1+\frac{\epsilon}{2}))
		\]
		Inspecting the ratio
		\[
		\frac{\sum_{i=n+1}^{n(1+\epsilon)} f(i)}{\sum_{i=1}^{n} f(i)}>\frac{n\frac{\epsilon}{2} f(n(1+\frac{\epsilon}{2}))}{nf(n)}>\frac{\epsilon}{2}\frac{f(n(1+\frac{\epsilon}{2}))}{f(n)}
		\]
		Since $f$ is rapidly varying function
		\[
		\lim_{n\rightarrow \infty} \frac{f(n(1+\frac{\epsilon}{2}))}{f(n)}=\infty
		\]
		So there exits $N$ such that
		\[
		\frac{f(n(1+\frac{\epsilon}{2}))}{f(n)}>\frac{2}{\epsilon}, n>N
		\]
	\end{proof}
	
	\begin{lemma}\label{lem:sumfAsymp}
		Let $f\in RV_\alpha$. The sum 
		\[\sum_{i=1}^{n} f(i)\]
		has asymptotic
		\[\frac{1}{1+\alpha}L(n)n^{\alpha+1}\]
	\end{lemma}
	
	\begin{proof}
		Using $f$'s asymptotic
		\[
		\sum_{i=1}^{n}f(i)\sim\sum_{i=1}^{n} i^{\alpha}L(i)
		\]
		The head does not affect the asymptotic
		\[
		\sim\sum_{i=n\epsilon}^{n} i^{\alpha}L(i)
		\]
		Since $L$ is SV, $\frac{L(\lambda x)}{L(x)}\to 1$ uniformly on compact sets. Take $\lambda\in[\epsilon, 1]$
		\[
		\sim\sum_{i=n\epsilon}^{n} i^{\alpha}L(n)
		\]
		Take the $L(n)$ out of the sum
		\[
		\sim L(n)\sum_{i=n\epsilon}^{n} i^{\alpha}
		\]
		Clearly
		\[
		\sim L(n)\int_{n\epsilon}^{n} t^{\alpha}dt
		\]
		Calculate the integral
		\[
		=\frac{1}{1+\alpha}L(n)n^{\alpha+1}(1-\epsilon^{1+\alpha})
		\]
		Follows
		\[
		\sim \frac{1}{1+\alpha}L(n)n^{\alpha+1}
		\]
	\end{proof}
	
	
	
	\begin{lemma}\label{lem:nSquaredIsLittle-o-sumInvF-case-a>1}
		Given monotonic regularly varying $f\sim x^{\alpha}L(x), \alpha>1$. Let $n$ be such that
		\[
		\sum_{i=1}^{n} f(i)\le x<\sum_{i=1}^{n+1} f(i)
		\]
		If  $\frac{L(xL^{\beta}(x))}{L(x)}\sim 1$ locally uniformly $\beta\in\mathbb{R}$ and $L$ is increasing and unbounded, then
		\[
		\sum_{k=1}^{n} f^{-1}(\frac{x}{k}))\sim C(\alpha)n^2
		\]
		Where $C(\alpha)=\frac{\alpha}{(\alpha-1)(\alpha+1)^{\frac{1}{\alpha}}}$
	\end{lemma}
	
	\begin{proof}
		From corollary 2.3.4 in \cite{bingham1987regular} it follows that
		\[
		f^{-1}(x)\sim x^{\frac{1}{\alpha}}\tilde{L}^{-\frac{1}{\alpha}}(x^{\frac{1}{\alpha}})\sim
		x^{\frac{1}{\alpha}}\frac{1}{L^{\frac{1}{\alpha}}(x^{\frac{1}{\alpha}})}
		\]
		Now we inspect
		\[
		\sum_{k=1}^{n} f^{-1}(\frac{x}{k})
		\]	
		Using $f^{-1}$ asymptotic
		\[
		\sim \sum_{k=1}^{n} (\frac{x}{k})^{\frac{1}{\alpha}}\frac{1}{L^{\frac{1}{\alpha}}(\frac{x}{k})}
		\]
		Take the $x$ out of the sum
		\[
		=x^{\frac{1}{\alpha}}\sum_{k=1}^{n} k^{-\frac{1}{\alpha}}\frac{1}{L^{\frac{1}{\alpha}}((\frac{x}{k})^{\frac{1}{\alpha}})}
		\]
		The head of sum does not affect the asymptotic
		\[
		\sim x^{\frac{1}{\alpha}}\sum_{k=n\epsilon}^{n} k^{-\frac{1}{\alpha}}\frac{1}{L^{\frac{1}{\alpha}}((\frac{x}{k})^{\frac{1}{\alpha}})}
		\]
		Since $L^{\frac{1}{\alpha}}(x^{\frac{1}{\alpha}})$ is SV, $\frac{L^{\frac{1}{\alpha}}((\lambda x)^{\frac{1}{\alpha}})}{L^{\frac{1}{\alpha}}(x^{\frac{1}{\alpha}})}\to 1$ uniformly on compact sets. Take $\lambda\in[\epsilon, 1]$
		\[
		\sim x^{\frac{1}{\alpha}}\sum_{k=n\epsilon}^{n} k^{-\frac{1}{\alpha}}\frac{1}{L^{\frac{1}{\alpha}}((\frac{x}{n})^{\frac{1}{\alpha}})}
		\]
		Take $\frac{1}{L^{\frac{1}{\alpha}}((\frac{x}{n})^{\frac{1}{\alpha}})}$ out of the sum
		\[
		\sim \frac{x^{\frac{1}{\alpha}}}{L^{\frac{1}{\alpha}}((\frac{x}{n})^{\frac{1}{\alpha}})}
		\sum_{k=n\epsilon}^{n} k^{-\frac{1}{\alpha}}
		\]
		Clearly
		\[
		\sim 
		\frac{x^{\frac{1}{\alpha}}}{L^{\frac{1}{\alpha}}((\frac{x}{n})^{\frac{1}{\alpha}})}
		\int_{n\epsilon}^{n} t^{-\frac{1}{\alpha}}dt
		\]
		Calculate the integral
		\[
		= \frac{1}{1-\frac{1}{\alpha}}
		\frac{x^{\frac{1}{\alpha}}}{L^{\frac{1}{\alpha}}((\frac{x}{n})^{\frac{1}{\alpha}})}
		(n^{1-\frac{1}{\alpha}}-(n\epsilon)^{1-\frac{1}{\alpha}})
		\]
		Factor n out
		\[
		=
		\frac{1}{1-\frac{1}{\alpha}}
		\frac{x^{\frac{1}{\alpha}}}{L^{\frac{1}{\alpha}}((\frac{x}{n})^{\frac{1}{\alpha}})}
		n^{1-\frac{1}{\alpha}}
		(1-\epsilon^{1-\frac{1}{\alpha}})
		\]
		We get
		\begin{equation}\label{eq:asympOfInvSum}
			\sim
			\frac{1}{1-\frac{1}{\alpha}}
			\frac{x^{\frac{1}{\alpha}}}{L^{\frac{1}{\alpha}}((\frac{x}{n})^{\frac{1}{\alpha}})}
			n^{1-\frac{1}{\alpha}}
		\end{equation}
		From lemma \autoref{lem:sumfAsymp}
		\begin{equation}\label{eq:xReln}
			x\sim\frac{1}{1+\alpha}L(n)n^{\alpha+1}
		\end{equation}
		Using \eqref{eq:xReln} to \eqref{eq:asympOfInvSum}
		\[
		\sum_{k=1}^{n} f^{-1}(\frac{x}{k})
		\sim
		\frac{1}{1-\frac{1}{\alpha}}\frac{(\frac{1}{1+\alpha}L(n)n^{\alpha+1})^{\frac{1}{\alpha}}}{L^{\frac{1}{\alpha}}((\frac{1}{1+\alpha}n^\alpha L(n))^{\frac{1}{\alpha}})}n^{1-\frac{1}{\alpha}}
		\]
		Simplify
		\[
		\sum_{k=1}^{n} f^{-1}(\frac{x}{k})
		\sim
		\frac{1}{1-\frac{1}{\alpha}}\frac{(\frac{1}{1+\alpha}L(n)n^{\alpha+1})^{\frac{1}{\alpha}}}{L^{\frac{1}{\alpha}}(\frac{1}{1+\alpha}n L(n)^{\frac{1}{\alpha}})}n^{1-\frac{1}{\alpha}}
		\]
		Using $\frac{L(xL^{\beta}(x))}{L(x)}\sim 1$ it follows
		\[
		\sum_{k=1}^{n} f^{-1}(\frac{x}{k})
		\sim
		\frac{1}{1-\frac{1}{\alpha}}\frac{(\frac{1}{1+\alpha}L(n)n^{\alpha+1})^{\frac{1}{\alpha}}}{L^{\frac{1}{\alpha}}(n)}n^{1-\frac{1}{\alpha}}
		\]
		Simplify
		\[
		\sum_{k=1}^{n} f^{-1}(\frac{x}{k})
		\sim
		\frac{\alpha}{(\alpha-1)(\alpha+1)^{\frac{1}{\alpha}}}
		n^{2}
		\]
	\end{proof}

	\begin{lemma}\label{lem:nSquaredIsLittle-o-sumInvF-case-a=1}
		Same assumptions as \autoref{lem:nSquaredIsLittle-o-sumInvF-case-a>1}, but $\alpha=1$ then 
		\[
		n^2=o(\sum_{k=1}^{n} f^{-1}(\frac{x}{k}))
		\]
	\end{lemma}
	
	\begin{proof}
		From proof of \autoref{lem:nSquaredIsLittle-o-sumInvF-case-a>1} we get
		\[
		\sum_{k=1}^{n} f^{-1}(\frac{x}{k})
		\sim 
		\frac{x}{L(x)}\int_{n\epsilon}^{n} \frac{1}{t}dt
		\sim
		\frac{x}{L(x)}\log(n)
		\]
		Using \autoref{lem:sumfAsymp} we get
		\[
		\sim
		n^2log(n)
		\]
		The claim follows.
	\end{proof}
	
	\begin{lemma}
		Let $\epsilon>0$ and $i$ such that $n\epsilon\le i$. Given $k,l\ge\epsilon, k-l\ge\epsilon$ and eventually monotonic $f\in RV_\alpha$
		\[
		\sum_{j=nl}^{nk} f(j)\sim \frac{L(n)}{\alpha+1}n^{\alpha+1}(k^{\alpha+1}-l^{\alpha+1})
		\]
	\end{lemma}
		
	\begin{proof}
		We will first prove that $k<l+C(\epsilon)$.
		\[
		\sum_{j=ln}^{kn} f(j)<n(k-l)f(kn)<n\epsilon f(kn)
		\]
		Using the sums asymptotic from \autoref{lem:sumfAsymp}
		\begin{align}
		1&\le\liminf\frac{n\epsilon f(kn)}{\frac{L(n)}{\alpha+1}n^{\alpha+1}}\notag\\
		1&\le\liminf\frac{(\alpha+1)\epsilon L(nk)k^\alpha}{L(n)}\notag\\
		1&\le\liminf(\alpha+1)\epsilon k^\alpha\notag\notag
		\end{align}
		Follows $k\le l+(\frac{1}{(\alpha+1)\epsilon})^{\frac{1}{\alpha}}$. Now since $l,k$ are in some compact interval independent of $n$, we can use uniform convergence theorem to get
		\begin{align}
			\sum_{j=nl}^{nk}f(j)&\sim \sum_{j=nl}^{nk}L(j)j^\alpha\notag\\
			&\sim \sum_{j=nl}^{nk} L(n)j^\alpha\notag\\
			&\sim \frac{L(n)}{\alpha+1}n^{\alpha+1}(k^{\alpha+1}-l^{\alpha+1})
		\end{align}	
	\end{proof}

	
	\begin{theorem}
		Let $f\in RV_\alpha, \alpha>1$, then $S(x)\sim C(\alpha)n^2$. Where 
		\[
		\sum_{i=1}^{n} f(i)\le x<\sum_{i=1}^{n+1} f(i)
		\] 
		and $C(\alpha)=\frac{1}{2(\alpha+1)}
		\frac{\Gamma(\frac{1}{\alpha+1})\Gamma(\frac{\alpha-1}{\alpha+1})}{\Gamma(\frac{\alpha}{\alpha+1})}$
	\end{theorem}
	
	\begin{proof}
		Let $l\ge\epsilon$. Define $x$ such that 
		\[
		\sum_{i=1}^{n} f(i)\le x<\sum_{i=1}^{n+1} f(i)
		\]
		Define $k$ such that
		\[
		\sum_{nl}^{nk} f(i)<x<\sum_{nl}^{nk+1} f(i)
		\]
		From \autoref{lem:sumfAsymp} and the fact that $\frac{k}{l}\in[1, \frac{1}{\epsilon}]$ we get using uniform convergence of $\frac{L(\lambda x)}{L(x)}\to1$ in compact sets
		\[
		\sum_{nl}^{nk} f(i)\sim \frac{L(n)}{\alpha+1}n^{\alpha+1}(k^{\alpha+1}-l^{\alpha+1})
		\] 
		On other hand we know that
		\[
		x\sim \frac{L(n)}{1+\alpha}n^{\alpha+1}
		\]
		Combining these
		\[
		\frac{L(n)}{1+\alpha}n^{\alpha+1}\sim\frac{L(n)}{\alpha+1}n^{\alpha+1}(k^{\alpha+1}-l^{\alpha+1})
		\]
		Follows
		\[
		k^{\alpha+1}-l^{\alpha+1}=1
		\]
		Solving for $k$
		\[
		k=(1+l^{\alpha+1})^{\frac{1}{\alpha+1}}
		\]
		Finally we get
		\begin{equation}\label{n(k-l) Asymp}
			k-l=(1+l^{\alpha+1})^{\frac{1}{\alpha+1}}-l
		\end{equation}
		Let $\phi(t)=(1+t^{\alpha+1})^{\frac{1}{\alpha+1}}-t$.
		Now we would like to count the $s_i(x)$. Calculating $s_i(x)$ asymptotic 
		Let $l$ be such that $i=n\phi(l)$. Solving for $nl$ we get
		\[
		nl=n\phi^{-1}(\frac{i}{n})
		\]
		So 
		\[
		s_i(x)= nl=n\phi^{-1}(\frac{i}{n})
		\]
		We can now count $S(x)$ asymptotic.
		\[
		S(x)=\sum_{i=1}^{n} s_i(x)
		\]
		The head doesn't affect the asymptotic
		%using the upperbound from upperbound lemma ti clearly follows%
		\[
		S(x)\sim\sum_{i=n\epsilon}^{n} s_i(x)
		\]
		Using $s_i$ formula
		\[
		S(x)\sim \sum_{i=n\epsilon}^{n} n\phi^{-1}(\frac{i}{n})
		\]
		Since $\phi$ is monotonic we have
		\[
		S(x)\sim n\int_{n\epsilon}^{n}\phi^{-1}(\frac{t}{n})dt
		\]
		Doing change of variables to $t\to un$ we get
		\[
		S(x)\sim n^2\int_{\epsilon}^{1} \phi^{-1}(t)dt
		\]
		Since $\phi:(0,\infty)\to(0,1)$ is strictly decreasing bijection %give details \phi and its properties. 
		\[
		\int_{\epsilon}^{1} \phi^{-1}(t)dt\sim\int_{0}^{\infty} \phi(t)dt
		\]
		We get
		\[
		S(x)\sim n^2\int_{0}^{\infty} \phi(t)dt
		\]
		Calculating the integral
		\[
			\int_{0}^{\infty} (1+x^{\alpha+1})^{\frac{1}{\alpha+1}}-xdx
		\]
		Doing change of variables $t=\frac{x^{\alpha+1}}{1+x^{\alpha+1}}$, $x=(\frac{t}{1-t})^{\frac{1}{\alpha+1}}$. We get
		\[
		dx=\frac{1}{\alpha+1}\big(\frac{t}{1-t}\big)^{-\frac{\alpha}{\alpha+1}}(1-t)^{-2}dt
		\]
		Changing $\phi(x)$ to $t$ terms \\
		\begin{align}
		\phi(x)&=(1+x^{\alpha+1})^{\frac{1}{\alpha+1}}-x \notag \\
		&=(\frac{1}{1-t})^{\frac{1}{\alpha+1}}-(\frac{t}{1-t})^{\frac{1}{\alpha+1}} \notag	\\
		&=(1-t)^{-\frac{1}{\alpha+1}}(1-t^{\frac{1}{\alpha+1}})\notag
		\end{align}
		Putting it together
		\begin{align}
		\int_{0}^{\infty} \phi(t)dt&=\int_{0}^{1} (1-t)^{-\frac{1}{\alpha+1}}(1-t^{\frac{1}{\alpha+1}})
		\frac{1}{\alpha+1}\big(\frac{t}{1-t}\big)^{-\frac{\alpha}{\alpha+1}}(1-t)^{-2}dt\notag\\
		&=\frac{1}{\alpha+1}\int_{0}^{1} (1-t)^{\frac{\alpha}{\alpha+1}-\frac{1}{\alpha+1}-2}
		(t^{-\frac{\alpha}{\alpha+1}}-t^{\frac{1}{\alpha+1}-\frac{\alpha}{\alpha+1}})dt\notag\\
		&=\frac{1}{\alpha+1}\int_{0}^{1} (1-t)^{-\frac{\alpha+3}{\alpha+1}}
		(t^{-\frac{\alpha}{\alpha+1}}-t^{\frac{1-\alpha}{\alpha+1}})dt\notag\\
		&=\frac{1}{\alpha+1}\int_{0}^{1} (1-t)^{-\frac{\alpha+3}{\alpha+1}}t^{-\frac{\alpha}{\alpha+1}}dt-\frac{1}{\alpha+1}\int_{0}^{1} (1-t)^{-\frac{\alpha+3}{\alpha+1}}t^{\frac{1-\alpha}{\alpha+1}}dt\notag\\
		&=\frac{1}{\alpha+1}\lim_{\epsilon\to 0^+}\int_{0}^{1} (1-t)^{\epsilon-\frac{\alpha+3}{\alpha+1}}t^{-\frac{\alpha}{\alpha+1}}dt-\frac{1}{\alpha+1}\int_{0}^{1} (1-t)^{\epsilon-\frac{\alpha+3}{\alpha+1}}t^{\frac{1-\alpha}{\alpha+1}}dt\notag\\
		&=\frac{1}{\alpha+1}\lim_{\epsilon\to 0^+}\big(B(\frac{1}{\alpha+1}, \epsilon-\frac{2}{\alpha+1})-
		B(\frac{2}{\alpha+1}, \epsilon-\frac{2}{\alpha+1})\big)\notag
		\end{align}
		Where $B$ is the analytically extended beta function(see NIST DLMF § 5.12 \cite{NIST:DLMF})
		\[
		B(z_1, z_2)=\int_{0}^{1} t^{z_1-1}(1-t)^{z_2-1}dt=\frac{\Gamma(z_1)\Gamma(z_2)}{\Gamma(z_1+z_2)}
		\]
		Using gamma expression for beta functions we get
		\begin{align}
		\int_{0}^{\infty} \phi(t)dt&=\frac{1}{\alpha+1}\lim_{\epsilon\to 0^+}(
		\frac{\Gamma(\frac{1}{\alpha+1})\Gamma(\epsilon-\frac{2}{\alpha+1})}{\Gamma(\frac{1}{\alpha+1}+\epsilon-\frac{2}{\alpha+1})}
		-
		\frac{\Gamma(\frac{2}{\alpha+1})\Gamma(\epsilon-\frac{2}{\alpha+1})}{\Gamma(\frac{2}{\alpha+1}+\epsilon-\frac{2}{\alpha+1})}
		)\notag\\
		&=\frac{1}{\alpha+1}\lim_{\epsilon\to 0^+}(
		\frac{\Gamma(\frac{1}{\alpha+1})\Gamma(-\frac{2}{\alpha+1})}{\Gamma(-\frac{1}{\alpha+1})}
		-
		\frac{\Gamma(\frac{2}{\alpha+1})\Gamma(-\frac{2}{\alpha+1})}{\Gamma(\epsilon)}
		)\notag
		\end{align}
		Using $\Gamma(t)\sim\frac{1}{t}, t\to 0$ %needs source
		\[
		\lim_{\epsilon\to 0^+}\frac{\Gamma(\frac{2}{\alpha+1})\Gamma(-\frac{2}{\alpha+1})}{\Gamma(\epsilon)}=0	
		\]
		So the expression simplifies to
		\begin{align}
			\int_{0}^{\infty} \phi(t)dt&=\frac{1}{\alpha+1}
			\frac{\Gamma(\frac{1}{\alpha+1})\Gamma(-\frac{2}{\alpha+1})}{\Gamma(-\frac{1}{\alpha+1})}
			\notag\\
			&=\frac{1}{2(\alpha+1)}
			\frac{\Gamma(\frac{1}{\alpha+1})\Gamma(\frac{\alpha-1}{\alpha+1})}{\Gamma(\frac{\alpha}{\alpha+1})}
			\notag\\
			&=C(\alpha)\notag
		\end{align}
		Where second equality follows from $\Gamma(z+1)= z\Gamma(z)$.
		
		
	\end{proof}

	
	
	
	\bibliographystyle{plain}   % or try alpha, apalike, ieee, etc.
	\bibliography{references}   % name of your .bib file (no .bib extension)
	
\end{document}
