\documentclass{article}
\usepackage{amsmath, amssymb, amsthm}  % amssymb for \mathbb, amsthm for lemma
\usepackage[T1]{fontenc}
\usepackage[utf8]{inputenc}

\newtheorem{lemma}{Lemma}

\begin{document}
	
	
	\begin{lemma}
		Let $f:\mathbb{R}\to\mathbb{R}$ be a convex, twice differentiable function such that $f'(x) \ge 0$. Let $g:\mathbb{R}\to\mathbb{R}$ satisfy
		\[
		f(g(x)) \sim x
		\]
		Then
		\[
		g(x) \sim f^{-1}(x)
		\]
	\end{lemma}
	
	\begin{proof}
		From $f(g(x))\sim x$ follows
		\[
		f(g(x))=x+o(x)
		\]
		Taking inverse of $f$
		\[
		g(x)=f^{-1}(x+o(x))
		\]
		Since $f$ is convex and monotonic, we get
		\[
		\begin{split}
		f^{-1}((1+\epsilon)x)&\le(1+\epsilon)f^{-1}(x) \\
		f^{-1}((1-\epsilon)x)&\ge(1-\epsilon)f^{-1}(x)
		\end{split}
		\]
		Hence, 
		\[
		g(x)=f^{-1}(x+o(x))=(1+o(1))f^{-1}(x)\sim f^{-1}(x)
		\]
	\end{proof}
	
	\begin{lemma}
		Let $f$ be rapidly varying function. Fix $\epsilon>0$, then for sufficiently large $n$
		\[
		\sum_{i=1}^{n} f(i)<\sum_{i=n+1}^{n(1+\epsilon)} f(i)
		\]
	\end{lemma}
	
	\begin{proof}
		We will prove that the ratio tends to infinity. We examine the left-hand side
		\[
		\sum_{i=1}^{n}f(i)<nf(n)
		\]
		On the right-hand side
		\[
		\sum_{i=n+1}^{n(1+\epsilon)}f(i)>n\frac{\epsilon}{2} f(n(1+\frac{\epsilon}{2}))
		\]
		Inspecting the ratio
		\[
		\frac{\sum_{i=n+1}^{n(1+\epsilon)} f(i)}{\sum_{i=1}^{n} f(i)}>\frac{n\frac{\epsilon}{2} f(n(1+\frac{\epsilon}{2}))}{nf(n)}>\frac{\epsilon}{2}\frac{f(n(1+\frac{\epsilon}{2}))}{f(n)}
		\]
		Since $f$ is rapidly varying function
		\[
		\lim_{n\rightarrow \infty} \frac{f(n(1+\frac{\epsilon}{2}))}{f(n)}=\infty
		\]
		So there exits $N$ such that
		\[
		\frac{f(n(1+\frac{\epsilon}{2}))}{f(n)}>\frac{2}{\epsilon}, n>N
		\]
	\end{proof}
	
\end{document}
